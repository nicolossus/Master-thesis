%================================================================
\section{Objective of the Study}
%================================================================ 

The overall objective of this thesis is to investigate the ability and utility of simulation-based inference for identifying parameters in mechanistic models of neural dynamics. Specifically, we will investigate the performance of ABC using rejection sampling with post-sampling regression adjustment and the machine learning algorithm SNPE on two neuroscientific models; the Hodgkin-Huxley model \cite{HH1952} and the Brunel network model \cite{Brunel2000}. The primary focus of the study will be on ABC, and SNPE will be used mostly for comparison. 

The seminal Hodgkin-Huxley model is a biophysically detailed description of the ionic mechanisms underlying the initiation and propagation of action potentials in squid giant axons. We will assess the identifiability of the potassium and sodium conductance parameters by examining the width and location of the resulting posterior estimates. As many biophysically detailed neuron models use the Hodgkin-Huxley formalism, the original Hodgkin-Huxley model becomes an ideal case for assessing and illustrating the application of simulation-based inference. 

We also consider the Brunel network model for activity dynamics in local cortical networks. Much effort in computational neuroscience today concerns mechanistic models at the network level and the Brunel network is thoroughly analyzed in the literature. The Brunel network is a sparsely connected recurrent network consisting of one excitatory and one inhibitory population of leaky integrate-and-fire (LIF) neurons. The network may be in several different states of spiking activity, largely dependent on the values of the synaptic weight parameters. For the current investigation, we limit our analysis to two of these states; the synchronous regular (SR) state, where the neurons fire almost fully synchronized at high rates; and the asynchronous irregular (AI) state, where the neurons fire mostly independently at low rates. We will assess and compare the identifiability of the synaptic weight parameters with the network both in the SR and AI state. 

The choice of summary statistics is crucial for the performance of simulation-based inference algorithms, in particular ABC, as they need to constrain the model well. Therefore, we will also investigate summary statistics of spiking activity in detail. 

We divide the overall objective into six parts:
\begin{enumerate}
    \item Implement simulators for both the Hodgkin-Huxley and Brunel network model in Python. 
    \item Implement a general ABC rejection sampler with post-sampling regression adjustment in Python.
    \item Determine suitable summary statistics of the spiking activity using domain knowledge and develop or find methods for extracting them from the simulated neural data. 
    \item Assess how well the summary statistics constrain the model parameters by examining sensitivity through a correlation analysis. Based on the correlation analysis, implement an importance weighting procedure for the statistics. 
    \item Estimate the model parameter posteriors with both ABC and SNPE by using synthetic observed data generated by the simulators. 
    \item Compare the results obtained via ABC and SNPE and insights they might provide about the neuroscientific models. 
\end{enumerate}



