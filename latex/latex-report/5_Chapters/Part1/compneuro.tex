%================================================================
\chapter{Computational Neuroscience}\label{chap:compneuro}
%================================================================

\section{Modeling Electrical Activity in Neurons}

\subsubsection{The electric force on ions}

As ions are electrically charged they exert forces on and experience forces from other ions. The force acting on an ion is proportional to the ion’s charge, $q$. The \textit{electric field} at any point in space is defined as the force experienced by an object with a unit of positive charge. A positively charged ion in an electric field experiences a force acting in the direction of the electric field; a negatively charged ion experiences a force acting in exactly the opposite direction to the electric field. At any point in an electric field a charge has an \textit{electrical potential energy}. The difference in the potential energy per unit charge between any two points in the field is called the \textit{potential difference}, denoted $V$ and measured in volts.

\subsubsection{Cell membrane is a capacitor} 

That the cell membrane is a capacitor means there is a voltage difference $V$ across it, accompanied by equal positive ($Q+$) and negative ($Q-$) charges on each side of the neuron. 

A capacitor consists of two conductors separated by a non-conductive region. The non-conductive region can either be a vacuum or an electrical insulator material known as a dielectric. From Coulomb's law a charge on one conductor will exert a force on the charge carriers within the other conductor, attracting opposite polarity charge and repelling like polarity charges, thus an opposite polarity charge will be induced on the surface of the other conductor. The conductors thus hold equal and opposite charges on their facing surfaces, and the dielectric develops an electric field.

The strength of the electric field set up through the separation of ions between the plates of the capacitor is proportional to the magnitude of the excess charge $q$ on the plates. As the potential difference is proportional to the electric field, this means that the charge is proportional to the potential difference. The constant of proportionality is called the **capacitance** and is measured in **farads**. It is usually denoted by $C$ and indicates how much charge can be stored on a particular capacitor for a given potential difference across it:

$$ q = C V $$


%================================================================
\subsection{The Hodgkin-Huxley Model}\label{sec:hh}
%================================================================ 




