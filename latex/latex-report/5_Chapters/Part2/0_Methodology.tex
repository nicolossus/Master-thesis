%================================================================
\chapter{Methodology}\label{chap:methodology}
%================================================================

%================================================================
\section{Post-Processing of Posterior Samples}\label{sec:post_processing}
%================================================================

Post-sampling Adjustment

\alglanguage{pseudocode}
\begin{algorithm}[H]
\caption{Feed forward algorithm}
\label{alg:FeedForward}
\begin{algorithmic}[1] 

\For{sample $\mathbf{x}_j$ in data set}
        \State $\mathbf{a}^0 = \mathbf{x}_j$\;
\EndFor

\end{algorithmic}
\end{algorithm}

%================================================================
\subsection{Regression Adjustment}\label{sec:reg_adjust}
%================================================================

%================================================================
\section{HH Feature Extraction}\label{sec:hh_feature_extract}
%================================================================

%================================================================
\chapter{Computational Approach}\label{chap:computational}
%================================================================

%================================================================
\section{pyLFI}\label{sec:pylfi}
%================================================================




... example code ...

Most well-tested implementations will do a bit more than this under the hood, but the preceding function gives the gist of the expectation–maximization approach.

An ABC software should be flexible enough to accommodate the new developments of the field. Here, we introduce a generalist Python package \cw{pyLFI}. The price to pay for the generality and flexibility is that the simulation of data and the calculation of summary statistics are left to the users. 

%================================================================
\section{Method}\label{sec:Method}
%================================================================

%----------------------------------------------------------------
\subsection{Project Method 1}\label{sec:project method}
%----------------------------------------------------------------

% the LFI methods: used own implementations (pylfi) and well-managed Python packages (ABCpy, sbi). => one goal is to see how they compare
% optimization beyond the scope of this thesis => use the established tools for the final (complex) analyses. 