%================================================================
\chapter{Summary \& Conclusions}\label{chap:Conclusion}
%\chapter{Conclusion \& Future Research}\label{chap:Conclusion}
%================================================================

%----------------------------------------------------------------
\section{Summary}\label{sec:summary}
%----------------------------------------------------------------

Summarize main findings. Connect results to the literature. 

%----------------------------------------------------------------
\section{Conclusion}\label{sec:conclusion}
%----------------------------------------------------------------

Concluding remarks.

A limiting factor of the ABC methods are that they are simulation intensive and the efficiency of the procedure depends on the the computational demands of the simulator.  
ABC has been successfully applied to a wide range of problems with an complex or absent associated likelihood. There are, however, several pitfalls to be aware of. The approach is simulation intensive, requires tuning of the tolerance threshold, discrepancy function and weighting function, and suffers from a curse of dimensionality of the summary statistic.  

%----------------------------------------------------------------
\section{Future Research}\label{sec:future}
%----------------------------------------------------------------

Possible future studies. Reference other relevant studies. 

Use more refined ABC methods
(SMC ABC ++, SNPE and SNL)

Compare with principled modelling; thermodynamic models

Metamodelling

Scalability 

Check if is scalable in the number of parameters;
Higher dimensional problems (more parameters to estimate)

