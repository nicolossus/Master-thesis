The amount of inhibition determines the amount of recurrent input in the network. Typically

absence of external drive together with the highly irregular spiking of individual

From a biophysical point of view, this is not entirely surprising. The amount of inhibition determines the amount of recurrent input in the network 

Asynchronous irregular (AI) activity is a hallmark of recurrent networks in which excitation is balanced by inhibition, and can persist even in the absence of external noise sources

network state in which both spikes and bursts occur asynchronously and irregularly at low rate.

reccurent input tends to dominate, thus diminishing the role of the external drive

As such, we expect this set of summary statistics to constrain $g$ better than $\eta$.

the inhibition determines the amount of recurrent input in the network, 

is to characterize the effects of parameter variability on the output of the model in terms of the summary statistics.


---

Due to the random nature of spiking activity, there can be a large variability in observed dynamics

stochastic spike generation.
---

In experiments, however, when the exactly same stimulus is repeated several times, the voltage traces elicited differ among themselves to a significant degree. 

Since the target data traces themselves are variable and selecting but one of the traces must to some extent be arbitrary, a direct trace to trace comparison between single traces might not serve as the best method of comparison between experiment and model. Indeed, this intrinsic variability ("noise") may have an important functional role. 

Therefore, we propose extracting certain features of the voltage response to a stimulus along with their intrinsic variability rather than using the voltage trace itself directly. 