%================================================================
%------------------------- Abstract -----------------------------
%================================================================
\chapter*{Abstract}
\addcontentsline{toc}{chapter}{Abstract}
\thispagestyle{plain}

Limit the abstract to as few words as possible. The abstract should always be less than one page long, and less than 400 words. Be aware that many online referencing systems only allow the first 200 words to be included. No figures or references should be presented.Avoid extensive technical and method details where possible. Should be readable to a literate science reader familiar with your general area, but not necessarily experts-only material.

Mechanistic models in neuroscience aim to explain neural or behavioral phenomena in terms of causal mechanisms, and candidate models are validated by investigating whether proposed mechanisms can explain how experimental data manifests. The mechanistic modelling is generally through the use of differential equations, and these models often have non-measurable parameters. A central challenge in building a mechanistic model is to identify the parametrization of the system which achieves an agreement between the model and experimental data.

The Bayesian paradigm of statistical inference provides a robust approach to parameter identification with quantified uncertainty. Statistical inference uses the likelihood function to quantify the match between parameters and data by deriving estimators of the parameters from the data. In Bayesian inference, posterior beliefs about parameters are updated according to Bayes' theorem upon observing data.

Many mechanistic models are defined implicitly through simulators, i.e. a set of dynamical equations, which can be run forward to generate data. Likelihoods can be derived for purely statistical models, but are generally intractable or computationally infeasible for simulation-based models. Hence are traditional methods in the toolkit of statistical inference inaccessible for many mechanistic models.

To overcome intractable likelihoods, a suite of methods that bypass the evaluation of the likelihood function, called likelihood-free inference methods, have been developed. These methods seek to directly estimate either the posterior or the likelihood, and require only the ability to generate data from the simulator to analyze the model in a fully Bayesian context.