%================================================================
%------------------------- Abstract -----------------------------
%================================================================
\chapter*{Abstract}
\addcontentsline{toc}{chapter}{Abstract}
\thispagestyle{plain}

%Limit the abstract to as few words as possible. The abstract should always be less than one page long, and less than 400 words. Be aware that many online referencing systems only allow the first 200 words to be included. No figures or references should be presented.Avoid extensive technical and method details where possible. Should be readable to a literate science reader familiar with your general area, but not necessarily experts-only material. 


A central challenge in building a mechanistic model of neural dynamics is to identify the model parameters consistent with experimental data. Due to intractable likelihoods, traditional methods in the toolkit of statistical inference are inaccessible for many mechanistic models. To overcome intractable likelihoods, simulation-based inference provides a framework for performing rigorous Bayesian inference without requiring numerical evaluation of likelihoods by using only forward simulations. The objective of this thesis is to investigate the viability of simulation-based inference, in particular approximate Bayesian computation (ABC) algorithms, for identifying parameters in mechanistic models of neural dynamics. Specifically, we use rejection ABC (REJ-ABC) and Markov chain Monte Carlo ABC (MCMC-ABC) to infer the conductance parameters in the Hodgkin-Huxley model for initiation and propagation of action potentials and the synaptic weight parameters in the Brunel network model for activity dynamics in local cortical networks. 

It also has the
advantage of returning an entire posterior distribution on the
value of the parameters, rather than a simple point estimate

implement the generic Python library pyLFI 

We implemented REJ-ABC with quantile-based rejection

We extensively varied hyperparameters, and found that

As the curse of dimensionality forces ABC to require a compression of data into low-level summary statistics, we use expert-crafted statistics of spiking activity. 



While rejection ABC (REJ-ABC) uses the prior as a proposal distribution, the efficiency can be improved by using sequentially refined proposal distributions (SMC) (can be said about MCMC also?). We implemented REJ-ABC with quantile-based rejection. 
We extensively varied hyperparameters. We investigated linear regression adjustment (Blum and François, 2010) and the summary statistics approach
by Prangle et al. (2014). [SBI Benchmark]



Because of the curse of dimensionality, ABC requires a compression of the data into low-dimensional summary statistics. We use various summary statistics of neural data obtained from domain knowledge.  expert-crafted summary statistics of spiking activity. If powerful low-dimensional summary statistics are established, traditional techniques can still offer a reasonable performance.

MCMC ABC, which improves the sample efficiency compared to Rej ABC by being guided by a proposal distribution ... MCMC ABC has improved sample efficiency 



In this thesis, we have used ABC with rejection sampling (Rejection ABC) and

In this thesis we implement the generic Python library pyLFI which uses ABC with both rejection and Markov chain Monte Carlo (MCMC) sampling for parameter inference/identification. In particular, we infer the conductance parameters in the Hodgkin-Huxley model for initiation and propagation of action potentials and the synaptic weight parameters in the Brunel network model for activity dynamics in local cortical networks. 

In this thesis, we have used (rejection) ABC (rejection sampling) and (MCMC) ABC (importance sampling) for identifying the conductance parameters in the Hodgkin-Huxley model for initiation and propagation of action potentials and the synaptic weight parameters in the Brunel network model for activity dynamics in local cortical networks. 

The Hodgkin-Huxley formalism is used as basis for many biophysically detailed cell models ...

Many cell models are built on the Hodgkin-Huxley formalism ... so being able to accurately constrain model parameters

using Markov chain Monte Carlo (MCMC) sampling

We were able to accurately constrain the conductance parameters of the Hodgkin-Huxley model as we obtained narrow posteriors with the ground truth parameters in regions of high posterior density. A 
For the Hodgkin-Huxley model we obtain narrow posteriors with the true parameters in regions of high posterior density, indicating that ABC is a viable method for parameter identification in the many cell models built on the Hodgkin formalism. 

As we only investigated models with few parameters, the viability of ABC on high-dimensional problems is yet to explore. 


REJ-ABC with local linear regression adjustment were able to constrain the potassium conductance $\bar{g}_\mathrm{K}$ to the credible interval ... (ground truth $\bar{g}_\mathrm{K} = 36$ and the sodium conductance $\bar{g}_\mathrm{Na}$ to the credible interval ... (ground truth $\bar{g}_\mathrm{Na} = 120$).

ABC requires some tuning to achieve different levels of accuracy, which can be time consuming as neural models generally are computationally expensive. 